    \section{Overview}
    The UserDemos repository is created to allow development on SLIMpy without breaking any users code.

    There are three main branches:
	\begin{definitions} 
    	\term{UserDemos/TRUNK}
    	is for the develoment of the user and you may store broken code in this repository.
    	\term{UserDemos/branches/beta}
    	allows the user to check in releases that work with 
            the current stable release of slimpy.
            When a user updates the beta branch we will guaranty that 
            if the code works with the current stable SLIMpy release then we will not break it with
            with the next stable release
        
    	\term{UserDemos/branches/stable}
            The stable branch will be updated once a developer aproves of the changes to the beta release. 
	\end{definitions}

    \section{Example}
    
    \begin{notice} [note] 
    if you are creating a new directory you must do 'svn copy' not 'svn merge'! 
    \end{notice}
    \begin{enumerate}
    \item
    from your package directory commit your latest revision
    
    'source howto.sh'
    export these variables to the repository  and package you want to update
\item
    dir of the repo on your machine (can be relative or absolute path)
    
    export local_dir="." 
    dir of the package you want to update
    
    export update_package="/sross/simple" 
    
\item
    switch to the beta branch by using 
    
    'ud_switch'
    
    merge your dev package with the beta branch by using 
    
    'ud_merge'
    
    
\item
commit the changes of the merge by doing
\begin{notice} [warning] 
    TEST that the megrge worked by running your demmo.
\end{notice} 
    
    \textbf{svn commit}
\item
    switch back to the dev branch by using 'ud_switch_back' 
    \end{enumerate}

