\begin{abstract}

Inverse problems in (exploration) seismology are known for their large
to very large scale. For instance, certain sparsity-promoting
inversion techniques involve vectors that easily exceed $2^{30}$ unknowns
while seismic imaging involves the construction and application of
matrix-free discretized operators where single matrix-vector
evaluations may require hours, days or even weeks on large compute
clusters. For these reasons, software development in this field has
remained the domain of highly technical codes programmed in low-level
languages with little eye for easy development, code reuse and
integration with (nonlinear) programs that solve inverse problems.

Following ideas from the Symes' \citetitle[http://www.trip.caam.rice.edu/txt/tripinfo/rvl.html]{Rice Vector Library} and Bartlett's
C++ object-oriented interface, \citetitle[http://software.sandia.gov/trilinos/packages/thyra/]{Thyra}, and Reduction/Transformation
operators (both part of the \citetitle[http://software.sandia.gov/trilinos]{Trilinos} software package), we developed a
software-development environment based on overloading. This
environment provides a pathway from in-core prototype development to
out-of-core and MPI 'production' code with a high level of code
reuse. This code reuse is accomplished by integrating the out-of-core
and MPI functionality into the dynamic object-oriented programming
language Python. This integration is implemented through operator
overloading and allows for the development of a coordinate-free solver
framework that (i) promotes code reuse; (ii) analyses the statements
in an abstract syntax tree and (iii) generates executable statements.

In the current implementation, we developed an interface to generate
executable statements for the out-of-core unix-pipe based (seismic)
processing package RSF-Madagascar (rsf.sf.net). The modular design
allows for interfaces to other seismic processing packages and to
in-core Python packages such as numpy.

So far, the implementation overloads linear operators and element-wise
reduction/transformation operators. We are planning extensions towards
nonlinear operators and integration with existing (parallel) solver
frameworks such as Trilinos.



\end{abstract}