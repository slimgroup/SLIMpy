\documentclass{manual}



\author{Sean Ross-Ross }

\title{SLIMpy}

\makeindex

\authoraddress{
        Seismic Laboratory for Imaging and Modeling  \\
        Email: \email{srossross@gmail.com}
}

\begin{document}

\maketitle

%%%%%%%%%%%%%%%%%%%%%%%%%%%%%%%%%%%%%%%%%%%%%
%%%%%%%%%%%%%%%%%%%%%%%%%%%%%%%%%%%%%%%%%%%%%

\ifhtml
\chapter*{Front Matter \label{front}}
\fi

\begin{definitions}

\term{ User Mailing List } 
\url{http://slim.eos.ubc.ca/mailman/listinfo/slimpy-user}


\term{ Developer Mailing List }
\url{http://slim.eos.ubc.ca/mailman/listinfo/slimpy-devel}


\term{Emails}
Email \citetitle[mailto:nadmin@slim.eos.ubc.ca]
    {nadmin@slim.eos.ubc.ca}
 for requests to join the developer team


\end{definitions}

\begin{abstract}

Inverse problems in (exploration) seismology are known for their large
to very large scale. For instance, certain sparsity-promoting
inversion techniques involve vectors that easily exceed $2^{30}$ unknowns
while seismic imaging involves the construction and application of
matrix-free discretized operators where single matrix-vector
evaluations may require hours, days or even weeks on large compute
clusters. For these reasons, software development in this field has
remained the domain of highly technical codes programmed in low-level
languages with little eye for easy development, code reuse and
integration with (nonlinear) programs that solve inverse problems.

Following ideas from the Symes' \citetitle[http://www.trip.caam.rice.edu/txt/tripinfo/rvl.html]{Rice Vector Library} and Bartlett's
C++ object-oriented interface, \citetitle[http://software.sandia.gov/trilinos/packages/thyra/]{Thyra}, and Reduction/Transformation
operators (both part of the \citetitle[http://software.sandia.gov/trilinos]{Trilinos} software package), we developed a
software-development environment based on overloading. This
environment provides a pathway from in-core prototype development to
out-of-core and MPI 'production' code with a high level of code
reuse. This code reuse is accomplished by integrating the out-of-core
and MPI functionality into the dynamic object-oriented programming
language Python. This integration is implemented through operator
overloading and allows for the development of a coordinate-free solver
framework that (i) promotes code reuse; (ii) analyses the statements
in an abstract syntax tree and (iii) generates executable statements.

In the current implementation, we developed an interface to generate
executable statements for the out-of-core unix-pipe based (seismic)
processing package RSF-Madagascar (rsf.sf.net). The modular design
allows for interfaces to other seismic processing packages and to
in-core Python packages such as numpy.

So far, the implementation overloads linear operators and element-wise
reduction/transformation operators. We are planning extensions towards
nonlinear operators and integration with existing (parallel) solver
frameworks such as Trilinos.



\end{abstract}

%%%%%%%%%%%%%%%%%%%%%%%%%%%%%%%%%%%%%%%%%%%%%
%%%%%%%%%%%%%%%%%%%%%%%%%%%%%%%%%%%%%%%%%%%%%

\section { License }

\verbatiminput{../license.doc}


%%%%%%%%%%%%%%%%%%%%%%%%%%%%%%%%%%%%%%%%%%%%%
%%%%%%%%%%%%%%%%%%%%%%%%%%%%%%%%%%%%%%%%%%%%%
\chapter{ What Is SLIMpy? }

SLIMpy is a software library that contains definitions of coordinate free vectors and  linear
operators. The SLIMpy library allows the user to design and run algorithms with any seismic software 
% explain o-o-c
package such as Madagascar \textbf{CITE?}, SepLib \textbf{CITE?} and SU \textbf{CITE?}. 
The common feature of these packages is to use out-of-core 
algorithms, that is i.e, algorithms that process  
to much data to fit it all into the computers main memory. SLIMpy allows the user to design an run algorithms using a Matlab \textbf{CITE?} 
style interface while executing each command with a seismic software package.
SLIMpy allows for flexible and fast prototyping of algorithms.
SLMpy extends each out-of-core package by
providing a parallel IO service.

SLIMpy looks at each main program of each of these out-of-core packages as a matrix vector
operation or vector reduction/transformation operation. It uses operator
overloading to generate an abstract syntax tree (AST) which can be optimized in
many ways before executing its commands. The AST also provides a pathway for
embarrassingly parallel and multi-process applications by sending branches of 
the tree over different nodes and processors. 
SLIMpy provides an interface to these out-of-core packages that allows optimal
construction of commands and allows for iterative techniques such as conjugate gradients or
$l_{1}$ solvers.
As such, SLIMpy smoothes the
transition from other languages such as Matlab and allows the algorithm designer
to write readable and reusable code. 

Inverse problems in seismology are large. The number of
unknowns can  exceed $2^{30}$, matrix-vector operations can take hours, days, or
even weeks. The software development cycle emphasizes processing flows which
typically involve highly technical coding, little code reuse and poor readability. 

Out-of-core-memory software
environments dominate the  geophysical processing field.
Such environments are designed to handle data that does not fit in main memory. Each base
operation is created as a main program that reads  and writes data from the hard disk. 
The main programs can also be chained together on stdin/out pipes
using a shell, writing data to disk only at the end of the pipe. 
To be efficient, an algorithm using an out-of-core package
 must have its operations chained together into the
longest pipe avoiding excessive disk I/O .
As a result, designing an I/O efficient algorithm can be very complex, especially 
when iterative methods are used. 
These algorithms
tend to be written in shell scripts and can therefor be difficult to read, understand and maintain.  

The research at SLIM utilizes non-separable transforms such
 as the curvelet %\cite{FDCT}
  or surfacelet %\cite{ndfb_surf} 
  transform. 
 The use of non-separable transforms coupled with the size of the problems 
 prevent development and use of iterative algorithms at a low level such
as C, C++ or Fortran. In some cases, such as the curvelet transform, the memory requirements
restrict the number of operations to one curvelet transform per processor. Writing
an $l_{1}$ solver in c-shell or bash was becoming very difficult and cumbersome.



\chapter{ Download and Install }

    

%%%%%%%%%%%%%%%%%%%%%%%%%%%%%%%%%%%%%%%%%%%%%

%%%%%%%%%%%%%%%%%%%%%%%%%%%%%%%%%%%%%%%%%%%%%

\par{Prerequisites}

\begin{definitions}

	\term{Python} SLIMpy is written in pure Python. python 2.5 is required 
	\url{http://www.python.org}
	
	\term{SCons} is a build tool similar to make. you will need version 0.98.x and
	up.
 	\url{http://www.scons.org}
	
	\term{svn} is a versioning client similar to cvs \url{http://www.subversion.tigris.org}
	
	\term{Madagascar} is an open-source software package for geophysical data
	processing and reproducible numerical experiments.
	\url{http://rsf.sourceforge.net/}
	
	
\end{definitions}

%%%%%%%%%%%%%%%%%%%%%%%%%%%%%%%%%%%%%%%%%%%%%
%%%%%%%%%%%%%%%%%%%%%%%%%%%%%%%%%%%%%%%%%%%%%

\par{Acquiring SLIMpy }


To install slimpy check it out of the slimpy repository by using the command 

\samp{\program{svn} \programopt{co}
\programopt{https://wave.eos.ubc.ca/Public/Public.Software.SLIMpy/STABLE}
\programopt{./core}} 

\par{Installing SLIMpy }

Once the files have downloaded onto your machine change
in to the \file{core} directory and type

\samp{\program{python} \program{setup.py} \programopt{install}
[\longprogramopt{prefix=/path/to/mypython}]}


\begin{notice} [note] 
If you use the \longprogramopt{prefix} option the path you set must be on your \envvar{PYTHONPATH}.
\end{notice}

%%%%%%%%%%%%%%%%%%%%%%%%%%%%%%%%%%%%%%%%%%%%%
%%%%%%%%%%%%%%%%%%%%%%%%%%%%%%%%%%%%%%%%%%%%%

%%%%%%%%%%%%%%%%%%%%%%%%%%%%%%%%%%%%%%%%%%%%%

\chapter{Getting Started}

	

	\section{Overview}
		This section is for those who want to program a SLIMpy ANA or Application.
		
		SLIMpy contains Vectors, vector spaces and  linear operators. 
		
		%TODO: doc
		More documentation to come
		
		
	\section{Configuring SLIMpy}
	   
	   SLIMpy contains no architecture dependent code but it does run into problems running other 
	   code. that is why you can and should configure SLIMpy. 
	   
	   to configure SLIMpy, run the command 
	   \code{configure_slimpy} located in the SLIMpy/scripts directory. this will write a file 
	   \code{~/.slimpy_rc}
	   witch is a python file that contains the default variables for SLIMpy.
	   
   Logging options:
   \begin{tableiii}{l|l|l}{exception}{Variable Name}{Default}{Info}
		\lineiii{verbose}{0}{set the verbosity from 1 to 10}
		\lineiii{abridge}{True}{shorten output with abridge-map. replace key with value
		for key/value pairs in abridge-map option}
		\lineiii{logfile}{None}{file to print to}
		\lineiii{debug}{['cmd','err']}{print extra info to logfile eg. print >>
		log(10,"foo") will print if debug=foo}
   \end{tableiii}

Distributed/MPI Options:       
    \begin{tableiii}{l|l|l}{exception}{Variable Name}{Default}{Info}
		\lineiii{NODELIST}{[]}{python list containing node names for mpi and
		distributed SLIMpy}
		\lineiii{NODEFILE}{PBS_NODEFILE}{file containing list of node names for mpi}
		\lineiii{localtmpdir}{/var/tmp}{used in mpi - temp header files will be built (local to each node)}
		\lineiii{np}{1}{number of processors running in mpi}
		\lineiii{mpi}{False}{ deprecated }
		\lineiii{rsh}{ssh}{distributed mode: command used to talk with nodes}
		\lineiii{MPIRUN}{mpirun_ssh}{ mpi executable}
		\lineiii{MPIFLAGS}{MPIRUN -np np -hostfile NODEFILE}{flags to pass to mpi on the command line}
   \end{tableiii}

Other Options:
\begin{tableiii}{l|l|l}{exception}{Variable Name}{Default}{Info}

	\lineiii{run_again}{False}{Run a command again upon failing}
	\lineiii{memsize}{500}{set the memory available to each program}
	\lineiii{sync_disk_om_rm}{True}{calls SYNC_CMD before removing data on disk}
	\lineiii{globaltmpdir}{HOME}{mpi - temp header files will be built - global for every node }
	\lineiii{datapath}{/Tools/toolboxes/rsf_stuff/datapath}{path where permanent data files will be built}
	\lineiii{test_devel}{False}{run developer tests, usually known bugs}
	\lineiii{use_scalar_obj}{True}{Track scalars in AST with slimpy scalar object}
	\lineiii{no_del}{False}{bool, if true SLIMpy will not delete data until the end}
	\lineiii{check_path}{False}{bool - check if all executable paths exist}
	\lineiii{runtype}{normal}{"normal" or "dryrun"}
	\lineiii{walltime}{60*30}{amount of time a command is allowed to take}
	\lineiii{SYNC_CMD}{sync}{No Doc}
	\lineiii{strict_check}{2}{0,1 or 2. On 0 no domain/range checking. 1 checking. 2 no void Spaces will be accepted}
	\lineiii{keep_tb_info}{True}{No Doc}

\end{tableiii} 

Obsolete Options:
\begin{tableiii}{l|l|l}{exception}{Variable Name}{Default}{Info}
	\lineiii{show_std_err}{False}{Replaced by debug option}
	\lineiii{tmpdatapath}{/var/tmp}{No Doc}
	\lineiii{nwin}{0}{dimensions to split the data into - mpi}
	\lineiii{eps}{0}{Domain decomposition overlap}

\end{tableiii} 

\chapter{Tutorials}
	
	Included are tutorials found in the directory SLIMpy/doc/tut/.
	If you are new to SLIMpy Look at the simple example.  
	\include{tutorials}

\chapter{Demos}

	\include{demos}

\chapter{Functionality}
    \citetitle[../funcs/index.html]{Functionality}
    % \include{functionality}

\chapter{Regression Tests}

    \include{tests}

\chapter{Reference}
    \section{Automatically generated reference}
    The SLIMpy reference guide was created with doxygen. 
    \citetitle[../ref/index.html]{SLIMpy Reference}
    \section{Man Page}
    \citetitle[../man/slimproj_man.html]{SLIMpy manpage}
    
    
\chapter{ FAQ }
	\section*{What does the \method{updateAbridgeMap} method do}
	\method{updateAbridgeMap} is a purely aesthetic function that replaces the output
	to make it more manageable For example the command:
\begin{verbatim} 
		>>> updateAbridgeMap(RSFROOT='')
\end{verbatim}
	will  replace all of the instances of
	'/User/someone/path/to/RSF/bin/sffunction' to 'sffunction'
	I find that is is much easier to work with that.

    \section*{Scons runs my demo again even when the result has been built}
    	if you import slimproj and use  builders such as \code{Solve} SLIMpy adds extra dependancies
	such as the SLIMpy svn revision number. to find out why scons is rebuilding your target run scons with the command line option \longprogramopt{debug=explain}
	
    \section*{How do I debug my SLIMpy script embedded in SCons}
    Learn the python interactive debugger!
    
    To stop python at a line and run the code interactively:
\begin{verbatim} 
		>>> import pdb
		>>> pdb.set_trace()
\end{verbatim}
On the command line you can do: scons \longprogramopt{debug=pdb}
 
or: 

In your SConstruct if you use a slim builder such as \code{Solve} you can add the 
parameter "post_mortem=True" to the builder method. This will allow you to inspect the code after it fails. 
    
    \section*{How can I know which linear operators are defined}
\begin{verbatim} 
		>>> import SLIMpy
		>>> SLIMpy.listLinop()
\end{verbatim}
    
    \section*{What is the difference between Scatter and ScatterMPI?}
    \method{Scatter} is a linear operator that wraps a number of windowing
    operators into a loop. The output of applying the scatter operator is an
    augmented vector. 
    
    \method{ScatterMPI} is a linear operator that uses an mpi command
    and the result is a "meta" vector - SLIM's xml data format - the "meta"
    format can be changed to an augmented vector and back with the \code{Meta} linear
    operator. If the error originates within SLIMpy's python code please pass me
    on the error. However if the error is being raised from a command line mpi
    subprocess call then your SLIMpy may not be configured properly. 
    
    See section 2.3 Configuring SLIMpy.
    
    
\end{document}
